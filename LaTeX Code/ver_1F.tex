\documentclass[document]{cup-journal}

\usepackage{amsmath}
\usepackage[nopatch]{microtype}
\usepackage{booktabs}
%\usepackage{lineno} %%%%%% switch when in draft 
%\linenumbers   %%%%%%%%%%%

\title{\emph{For the Love of Money}: NIMBY and the Anti-Government Funding Attitudes}

\author{Joe Blaszcynski}

\addbibresource{refs.bib}

\begin{document}

\begin{abstract}
Previous research overstates the important determinants of NIMBYs and their attitudes toward affordable housing. This research argues that individuals who share NIMBY sentiments and anti-government funding attitudes are one of the same. By utilizing a survey research design using the General Social Survey (GSS) dataset it studies the combined effect of NIMBY and anti-local government funding policies. Overall, the research echoes findings from previous research on individuals who are likely to be associated with the NIMBY movement, but does not find support for combining other anti-government funding attitudes. 
\end{abstract}

\vspace{-1.5cm}

\section*{Introduction} 

The Not in My Back Yard (NIMBY) movement is a stark anti-growth movement within cities. The people involved may not necessarily know or even recognize themselves to be involved in such a movement. However, often when patterns emerge regarding affordable housing in the local community, NIMBYs are the fierce advocates on the opposite side of the spectrum for that public policy. They tend to express their advocacy in local government meetings but NIMBYs tend to stay at home on election day \autocite{einstein2019participates}.  \\

Especially, to what effect does the NIMBY movement have on the broader local public policies? Moreover, these local government policy objectives have become unattainable because of groups such as the NIMBY movement that hold anti-growth senitment. If individuals who live in cities oppose growth for the city, then they will exhibit NIMBYS anti-growth attitudes for local government policy interests. Where creating affordable housing could improve economic conditions for the homeless but also improve economic incentives for the city \autocite{peterson1981city}. \\

Previous research has found that NIMBYs do not just oppose affordable housing but other developments such as daycare centers and nursing homes \autocite{dear1992understanding}. The core of the argument in sociology and urban studies has often been that NIMBYs are in opposition to development projects and that they and most likely are against supporting any other social policies. While also demonstrating a set of other


\printendnotes
\noindent Supplementary data available upon request. Replication code temporarily unavailable. 


\printbibliography
 

\end{document}
